\section{Instructions}\label{instructions-000}

\subsection{Exercise 1A. Run Pre-Defined Building with no Windows}\label{exercise-1a.-run-pre-defined-building-with-no-windows}

Objective:~ Learn to use EP-Launch to run an EnergyPlus input file and view output files.

1.~Open EP-Launch.

2.~Under ``Input File'', browse for input file Exercise1A.idf.~ This input file contains the 1-zone model described above without the windows and lights. This is located under the install folder \textless{}root\textgreater{}\textbackslash{}ExampleFiles\textbackslash{}BasicsFiles,

3.~Under ``Weather File'', select ``No Weather File'' (at the top of the pull-down list).

4.~Press ``Simulate''.

5.~When the simulation is complete, review output files:

\begin{itemize}
\item
  Press ``Text Output Files'' to see all text output.~ Look especially at the eio and err output files.
\item
  Press ``Drawing Files'' to see a dxf drawing of the building envelope.~ (If using Voloview Express, right-click to switch between wireframe and shaded orbit view.~ In DWG True View, use ``View'' -\textgreater{} ``Visual Styles'' to switch between wireframe and solid views. In both programs, use ``View'' à``Named Views'' to select isometric views.)
\item
  An empty svg drawing file will also open (this will show HVAC system components in later exercises).~ Note that the Adobe SVG viewer is a ``plug-in'' for Internet Explorer (IE), so IE will open when viewing an SVG file.~ Depending on the security settings in IE, you may be prompted with a warning about ``active'' content.
\item
  Press ``Spreadsheets'' to open the numeric csv output files.~ In Exercise1a.csv, review the pattern of outdoor conditions and loads.~ (To make it easier to read the column headings, select Row 1, format cells, and turn on wrap text; then select cell B2 and select ``freeze panes''.) ~In Exercise1aMeter.csv, review the facility district heating and cooling meters.
\item
  Zone/Sys Air Temperature -- the zone air temperatures are already being reported.
\item
  Outdoor Dry Bulb -- is being reported (so you can compare to outside temperature)
\item
  The meter for the heating in the facility - DistrictHeatingWater:Facility -- is being reported. Facility is the entire building.
\item
  The meter for the cooling in the facility - DistrictCooling:Facility -- is being reported.
\end{itemize}

\subsection{Exercise 1B. Add Windows}\label{exercise-1b.-add-windows}

Objective:~ Learn how to add materials, constructions, and a surface using 3-D coordinates.

1.~In EP-Launch, with input file Exercise1A.idf still selected, press ``Edit -- IDF Editor''.~ This will open Exercise1A.idf in the IDF Editor, a tool that assists in editing EnergyPlus input files (idf).

2.~In IDF Editor, select File -\textgreater{} Save Options . . . and set ``Saved Order'' to ``Original with New at Top'', and ``Special Format for Some Objects'' to ``Yes.''~ Check the ``Set as Default'' box.

3.~In IDF Editor, Select File -\textgreater{} Save As . . . and save this file as Exercise1B.idf.

4.~Create the construction definition for the windows which are double-pane clear gas with an air space:

\begin{itemize}
\item
  Using File -\textgreater{} Open Dataset, open the window glass materials dataset file, WindowGlassMaterials.idf
\item
  Scroll down the Class list and select ``\textbf{WindowMaterial:Glazing}''.
\end{itemize}

\begin{lstlisting}
-Hint:  In IDF Editor, View -> Show Classes with Objects Only (or ctl-L) will hide all empty object types from the class list.
\end{lstlisting}

\begin{itemize}
\item
  Locate the object which defines the material properties for ``CLEAR 6MM''.~ Select this object (by clicking on the column heading).
\item
  Using Edit -\textgreater{} Copy Object (or the toolbar button, or ctl-C), copy this object.
\item
  Switch windows to file Exercise1B.idf and paste the window material into this file.~ (Verify that is had been added by going to \textbf{WindowMaterial:Glazing} to view the object.)
\item
  Open dataset file WindowGasMaterials.idf.
\item
  Locate ``AIR 3MM'', copy it and paste it into Exercise1B.idf.
\item
  In Exercise1B.idf, select the ``\textbf{Construction}'' class.~ There are three constructions pre-defined for the walls, roof, and floor.
\item
  Press ``New Obj'' to create a new blank \textbf{Construction} object.
\item
  Name this new construction ``DOUBLE PANE WINDOW''.
\item
  Use the pulldown list to select ``CLEAR 6MM'' for the outside layer, then press ``Enter'' or ``Return'' to save this entry and move to the next field.
\item
  Select ``AIR 3MM'' for Layer 2, and ``CLEAR 6MM'' for Layer 3.
\end{itemize}

5.~Add the east window (3m wide by 2m high, centered on wall, \emph{see the drawing in} \emph{Figure~\ref{fig:schematic-for-exercise-1}} \emph{to determine coordinates):}

\begin{itemize}
\item
  Select ``\textbf{FenestrationSurface:Detailed}'' class.
\item
  Add a new object named ``EAST WINDOW''.
\item
  Set the remaining fields as listed:
\item
  Surface Type \ldots{}\ldots{}\ldots{}\ldots{}\ldots{}\ldots{}\ldots{}\ldots{}\ldots{}\ldots{}\ldots{}\ldots{}\ldots{}\ldots{}\ldots{}\ldots{} = Window
\item
  Construction Name \ldots{}\ldots{}\ldots{}\ldots{} = DOUBLE PANE WINDOW
\item
  Base Surface Name \ldots{}\ldots{}\ldots{}\ldots{}\ldots{} = ZONE SURFACE EAST
\item
  OutsideFaceEnvironment Object \ldots{}\ldots{}\ldots{}\ldots{}\ldots{}\ldots{}. = \textless{}blank\textgreater{}
\item
  View Factor to Ground \ldots{}\ldots{}\ldots{}\ldots{}\ldots{}\ldots{}\ldots{}\ldots{}.. = autocalculate
\item
  Name of shading control \ldots{}\ldots{}\ldots{}\ldots{}\ldots{}\ldots{}\ldots{}\ldots{}\ldots{}\ldots{} = \textless{}blank\textgreater{}
\item
  WindowFrameAndDivider Name \ldots{}\ldots{}\ldots{}\ldots{}\ldots{}\ldots{}.. = \textless{}blank\textgreater{}
\item
  Multiplier \ldots{}\ldots{}\ldots{}\ldots{}\ldots{}\ldots{}\ldots{}\ldots{}\ldots{}\ldots{}\ldots{}\ldots{}\ldots{}\ldots{}\ldots{}\ldots{}\ldots{}\ldots{}\ldots{}\ldots{}\ldots{}. = 1
\item
  Number of Surface Vertex Groups \ldots{}\ldots{}\ldots{}\ldots{}\ldots{}\ldots{}\ldots{}\ldots{}\ldots{} = 4
\item
  Vertex coordinates = \emph{as determined from the drawing} \emph{Figure~\ref{fig:schematic-for-exercise-1}}.~ Coordinates in this input are in World Coordinates (all relative to the global origin of 0,0,0).~ Coordinates are specified as viewed from the outside of the surface, using the rules specified in the SurfaceGeometry object.
\end{itemize}

6.~Add the west window, similar to the east window.

7.~Add a new \textbf{Output:Surfaces:List} object, type = Details.~ This report produces a list of all surfaces in the eio output summarizing area, azimuth, tilt, etc.

8.~Save and close the IDF file, select Exercise1B.idf in EP-Launch, run the simulation and view outputs.

\begin{itemize}
\item
  Always review the err file for errors and warnings.~ Fix problems if needed and re-run.
\item
  Are the windows in the right place in the dxf drawing file. (Use the Drawing File button or select the DXF file from View -\textgreater{} Single File or from the Quick-Open panel).
\item
  Review the surface details report in the eio file, search for ``Zone/Shading Surfaces'' to find this report. (Use the Text Output button, Quick Open ``eio'' button, or select from the single file menu, or use F7).~ This report is easier to read by pasting this section into a spreadsheet and using the text to columns function with comma as a delimiter).
\item
  Open the csv output file and compare the heating and cooling loads with the results from Exercise1A.csv.
\end{itemize}

\subsection{Exercise 1C. Add Internal Loads}\label{exercise-1c.-add-internal-loads}

Objective:~ Learn how to add schedules, internal loads, and report variables.

1.~Save Exercise1B.idf as Exercise1C.idf.

2.~Open the dataset file Schedules.idf:

\begin{itemize}
\item
  Copy the \textbf{Schedule:Compact} object named ``Office Lighting'', and paste it into Exercise1C.idf.
\item
  Copy the \textbf{ScheduleTypeLimits} object named ``Fraction'', and paste it into Exercise1C.idf.
\end{itemize}

3.~In Exercise1C.idf, add a LIGHTS object named ZONE ONE Lights, using the Office Lighting schedule, peak input is 1000W.~ Consult the EnergyPlus Input Output Reference section on \textbf{Lights} for values for the return, radiant, and visible fractions.~ Assume the lights are surface mounted fluorescents.

4.~Save and close the IDF file, select Exercise1C.idf in EP-Launch, run the simulation and review outputs.

5.~Open the rdd file (the report variable data dictionary) and find report variable names related to \textbf{Lights}.~ Add a new \textbf{Output:Variable} object to report the lighting electric consumption.

6.~Run the simulation and review outputs.

\begin{itemize}
\item
  Check the err file.
\item
  Find the lighting electric consumption in the csv output file.
\end{itemize}

7.~Compare heating and cooling loads with Exercise1A and Exercise1B.

8.~Add more \textbf{Output:Variable} objects as desired.

\subsection{Exercise 1D. Annual Simulation and Predefined Reports}\label{exercise-1d.-annual-simulation-and-predefined-reports}

Objective:~ Learn how to run an annual simulation using a weather data file and add table reports.

1.~Save Exercise1C.idf as Exercise1D.idf.

2.~Edit the \textbf{SimulationControl} object to turn off the design day simulations by setting ``Run Simulation for Sizing Periods'' to \textbf{No} and turn on the weather file (annual) simulation by setting ``Run Simulation for Weather File Run Periods'' to \textbf{Yes}..

3.~Add a RunPeriod object to run a full annual simulation, let other fields default or remain blank.

4.~Add a \textbf{Output:Table:SummaryReports} object, and select the following reports:~ ``Annual Building Performance Summary'' (ABUPS), ``Input Verification and Results Summary'' (IVRS), ``Climate Summary'', and ``Envelope Summary''.

5.~Add a \textbf{OutputControl:Table:Style} object, and select HTML format (ColumnSeparator).

6.~Edit existing \textbf{Output:Variable} and \textbf{Output:Meter} objects and change the reporting frequency from Hourly to Monthly.

7.~Save and close the IDF file, select Exercise1D.idf in EP-Launch.

8.~Select Chicago TMY2 weather file (or the weather file of your choice) and run the simulation.

9.~Review outputs.

\begin{itemize}
\item
  Check the err file.
\item
  Look at the monthly results in the csv output.
\item
  Press the Table output button to view the predefined reports.
\end{itemize}

\subsection{Solution: Exercise 1}\label{solution-exercise-1}

\emph{Try not to look at this section until you have completed the Exercise.}

\subsubsection{List of New Objects}\label{list-of-new-objects}

This is a listing of new and modified objects created in this Exercise.

\begin{lstlisting}
WindowMaterial:Glazing,
    CLEAR 6MM,               !- Name
    SpectralAverage,         !- Optical Data Type
    ,                        !- Name of Window Glass Spectral Data Set
    0.006,                   !- Thickness {m}
    0.775,                   !- Solar Transmittance at Normal Incidence
    0.071,                   !- Solar Reflectance at Normal Incidence: Front Side
    0.071,                   !- Solar Reflectance at Normal Incidence: Back Side
    0.881,                   !- Visible Transmittance at Normal Incidence
    0.080,                   !- Visible Reflectance at Normal Incidence: Front Side
    0.080,                   !- Visible Reflectance at Normal Incidence: Back Side
    0.0,                     !- IR Transmittance at Normal Incidence
    0.84,                    !- IR Hemispherical Emissivity: Front Side
    0.84,                    !- IR Hemispherical Emissivity: Back Side
    0.9;                     !- Conductivity {W/m-K}
\end{lstlisting}

\begin{lstlisting}
WindowMaterial:Gas,
    AIR 3MM,                 !- Name
    Air    ,                 !- Gas Type
    0.0032;                  !- Thickness {m}
\end{lstlisting}

\begin{lstlisting}
Construction,
    DOUBLE PANE WINDOW,      !- Name
    CLEAR 6MM,               !- Outside Layer
    AIR 3MM,                 !- Layer #2
    CLEAR 6MM;               !- Layer #3
\end{lstlisting}

\begin{lstlisting}
FenestrationSurface:Detailed,
    EAST WINDOW,             !- User Supplied Surface Name
    WINDOW,                  !- Surface Type
    DOUBLE PANE WINDOW,      !- Construction Name of the Surface
    ZONE SURFACE EAST,       !- Base Surface Name
    ,                        !- OutsideFaceEnvironment Object
    autocalculate,           !- View Factor to Ground
    ,                        !- Name of shading control
    ,                        !- WindowFrameAndDivider Name
    1,                       !- Multiplier
    4,                       !- Number of vertices
    8, 1.5, 2.35,            !- X,Y,Z  1 {m}
    8, 1.5, 0.35,            !- X,Y,Z  2 {m}
    8, 4.5, 0.35,            !- X,Y,Z  3 {m}
    8, 4.5, 2.35;            !- X,Y,Z  4 {m}
\end{lstlisting}

\begin{lstlisting}
FenestrationSurface:Detailed,
    WEST WINDOW,             !- User Supplied Surface Name
    WINDOW,                  !- Surface Type
    DOUBLE PANE WINDOW,      !- Construction Name of the Surface
    ZONE SURFACE WEST,       !- Base Surface Name
    ,                        !- OutsideFaceEnvironment Object
    autocalculate,           !- View Factor to Ground
    ,                        !- Name of shading control
    ,                        !- WindowFrameAndDivider Name
    1,                       !- Multiplier
    4,                       !- Number of Vertices
    0, 4.5, 2.35,            !- X,Y,Z  1 {m}
    0, 4.5, 0.35,            !- X,Y,Z  2 {m}
    0, 1.5, 0.35,            !- X,Y,Z  3 {m}
    0, 1.5, 2.35;            !- X,Y,Z  4 {m}
\end{lstlisting}

\begin{lstlisting}
Output:Surfaces:List,Details;
\end{lstlisting}

\begin{lstlisting}
Schedule:Compact,
    Office Lighting,         !- Name
    Fraction,                !- ScheduleType
    Through: 12/31,          !- Complex Field #1
    For: Weekdays SummerDesignDay,  !- Complex Field #2
    Until: 05:00, 0.05,      !- Complex Field #4
    Until: 07:00, 0.1,       !- Complex Field #6
    Until: 08:00, 0.3,       !- Complex Field #8
    Until: 17:00, 0.9,       !- Complex Field #10
    Until: 18:00, 0.5,       !- Complex Field #12
    Until: 20:00, 0.3,       !- Complex Field #14
    Until: 22:00, 0.2,       !- Complex Field #16
    Until: 23:00, 0.1,       !- Complex Field #18
    Until: 24:00, 0.05,      !- Complex Field #20
    For: Saturday WinterDesignDay,  !- Complex Field #21
    Until: 06:00, 0.05,      !- Complex Field #23
    Until: 08:00, 0.1,       !- Complex Field #25
    Until: 12:00, 0.3,       !- Complex Field #27
    Until: 17:00, 0.15,      !- Complex Field #29
    Until: 24:00, 0.05,      !- Complex Field #31
    For: Sunday Holidays AllOtherDays,  !- Complex Field #32
    Until: 24:00, 0.05;      !- Complex Field #34
\end{lstlisting}

\begin{lstlisting}
ScheduleTypeLimits,
    Fraction,                !- ScheduleType Name
    0.0,                     !- Lower Limit Value
    1.0,                     !- Upper Limit Value
    CONTINUOUS;              !- Numeric Type
\end{lstlisting}

\begin{lstlisting}
Lights,
    ZONE ONE Lights,         !- Name
    ZONE ONE,                !- Zone Name
    Office Lighting,         !- Schedule Name
    LightingLevel,           !- Design Level Calculation Method
    1000,                    !- Lighting Level {W}
    ,                        !- Watts per Zone Floor Area {W/m2}
    ,                        !- Watts per Person {W/person}
    0,                       !- Return Air Fraction
    0.72,                    !- Fraction Radiant
    0.18,                    !- Fraction Visible
    1,                       !- Fraction Replaceable
    General,                 !- End-Use Subcategory
    No;                      !- Return Air Fraction Calculated from Plenum Temperature
\end{lstlisting}

\begin{lstlisting}
Output:Variable,*,Lights Electric Consumption ,hourly;
\end{lstlisting}

\begin{lstlisting}
RunPeriod,
    1,                       !- Begin Month
    1,                       !- Begin Day Of Month
    12,                      !- End Month
    31,                      !- End Day Of Month
    UseWeatherFile,          !- Day Of Week For Start Day
    Yes,                     !- Use WeatherFile Holidays/Special Days
    Yes,                     !- Use WeatherFile DaylightSavingPeriod
    No,                      !- Apply Weekend Holiday Rule
    Yes,                     !- Use WeatherFile Rain Indicators
    Yes,                     !- Use WeatherFile Snow Indicators
    1;                       !- Number of years of simulation
\end{lstlisting}

\begin{lstlisting}
Output:Table:SummaryReports,
    Annual Building Utility Performance Summary,  !- ReportName1
    Input Verification and Results Summary,  !- ReportName2
    Climate Summary,         !- ReportName3
    Envelope Summary;        !- ReportName4
\end{lstlisting}

\begin{lstlisting}
OutputControl:Table,
    HTML;                    !- ColumnSeparator
\end{lstlisting}

\begin{lstlisting}
SimulationControl,
    No,                      !- Do the zone sizing calculation
    No,                      !- Do the system sizing calculation
    No,                      !- Do the plant sizing calculation
    No,                      !- Do the design day simulations
    Yes;                     !- Do the weather file simulation
\end{lstlisting}
